%----------------------------------------------------------------------------------------
%	PACKAGES AND DOCUMENT CONFIGURATIONS
%----------------------------------------------------------------------------------------
\documentclass{scrartcl}

%\usepackage{mathptmx}
\usepackage{graphicx} % Required for the inclusion of images
\usepackage{easybmat}
\usepackage{natbib} % Required to change bibliography style to APA
\usepackage{amsmath} % Required for some math elements 
\usepackage{amssymb}
\usepackage{float}
\usepackage{caption}
\usepackage{subcaption}


\usepackage{datetime}

\setlength\parindent{0pt} % Removes all indentation from paragraphs

\renewcommand{\labelenumi}{\alph{enumi}.} % Make numbering in the enumerate environment by letter rather than number (e.g. section 6)

%\usepackage{times} % Uncomment to use the Times New Roman font

%----------------------------------------------------------------------------------------
%	DOCUMENT INFORMATION
%----------------------------------------------------------------------------------------

\title{CS 390E: Geometry Processing \\ Assignment 02} % Title
\subtitle{Regression}
\author{Anna Fr\"{u}hst\"{u}ck} % Author name

\newdate{date}{05}{02}{2017}
%\date{\displaydate{date}}
\date{\today}
\begin{document}

\maketitle % Insert the title, author and date

%"C:\Program Files (x86)\SumatraPDF\SumatraPDF.exe" -reuse-instance -inverse-search "\"C:\Program Files (x86)\TeXstudio\texstudio.exe\" \"%%f\" -line %%l" "?am.pdf"

%----------------------------------------------------------------------------------------
%	SECTION 1
%----------------------------------------------------------------------------------------

\section{Linear Regression}
In linear regression, the assumption is that the model that generates the data involves a linear combination of the input variables.
$$y(x, w) = w_0 + w_1x_1 + ... + w_nx_n$$
or simplified 
$$y(x, w) = w_0 + \sum_{j=1}^{M-1}{w_jx_j}$$
For polynomial regression in one dimension, we 
$$y(x, w) = w_0 + \sum_{d=1}^{D}{w_dx^d}$$

\begin{figure}[H]
	\centering
	\begin{subfigure}{.45\textwidth}
		\centering
		\includegraphics[width=\linewidth,trim={2.5cm 2.5cm 2.5cm 2.5cm},clip]{img/regression_2dim}
		\caption{Polynomial of degree 2}
	\end{subfigure}%
	\begin{subfigure}{.45\textwidth}
		\centering
		\includegraphics[width=\linewidth,trim={2.5cm 2.5cm 2.5cm 2.5cm},clip]{img/regression_10dim_overfitting}
		\caption{Polynomial of degree 10}
		\label{fig:regression1overfitting}
	\end{subfigure}%
	\caption{
		Regression on data points generated along the quadratic polynomial $1.4x^2 + 0.3x - 0.8$ with added noise. Note the overfitting in Fig.~\ref{fig:regression1overfitting} when choosing a polynomial of a high degree.
	}
	\label{fig:regression1}
\end{figure}

\begin{figure}[H]
	\includegraphics[width=\linewidth,trim={2.5cm 1.5cm 2.5cm 1.5cm},clip]{img/regression_3_5_8dim}
	\caption{
		Regression on data points generated along a sine curve with added noise.\\ 
		{\footnotesize \textit{Polynomials: red = degree 3, yellow = degree 5, purple = degree 8.}}
		}
	\label{fig:regression2}
\end{figure}

\begin{figure}[H]
	\includegraphics[width=\linewidth,trim={2.5cm 1.5cm 2.5cm 1.5cm},clip]{img/polynomialfitting1}
	\caption{
		Iterative fitting of polynomial of degree $n-1$ to $n$ user-generated data points.
	}
	\label{fig:regression3}
\end{figure}

\section{Natural Cubic Splines}
\begin{figure}[H]
	\includegraphics[width=\linewidth,trim={2.5cm 1.5cm 2.5cm 1.5cm},clip]{img/mouse_splines}
	\caption{
		Natural cubic splines interpolating between points entered consecutively through mouse-click.
	}
	\label{fig:splines1}
\end{figure}
\begin{figure}[H]
	\includegraphics[width=\linewidth,trim={2.5cm 1.5cm 2.5cm 1.5cm},clip]{img/compare_splines2}
	\caption{
		{\footnotesize 
			\textit{blue solid = my implementation, \\
				red dotted = \emph{MATLAB \texttt{csape}} function,\\
				green dashed = \emph{MATLAB \texttt{spline}} function.}
		}\\
		Comparison of different implementations of cubic splines in MATLAB. Note the slightly different results due to differing end conditions.\\My implementation uses $P_0'(0)=0$ and $P_{n-1}'(1)=0$ as end conditions in this instance.
	}
	\label{fig:splines2}
\end{figure}
\begin{figure}[H]
	\includegraphics[width=\linewidth,trim={2.5cm 1.5cm 2.5cm 1.5cm},clip]{img/boundary_splines}
	\caption{
		Effects of differently chosen boundary conditions on spline generation.\\
		{\footnotesize 
			\textit{dotted line: first derivative boundary conditions $P_0'(0)=0$ and $P_{n-1}'(1)=0$\\
				dashed line: second derivative boundary conditions $P_0''(0)=0$ and $P_{n-1}''(1)=0$}
		}
	}
	\label{fig:splines3}
\end{figure}
\end{document}